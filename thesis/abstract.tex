%--------------------------------------------------------------------------
%% Abstract section.
\chapter*{Abstract\markboth{Abstract}{Abstract}}
\label{sec:abstract}

% short intro to the problem space and the opportunity

Modeling 3D objects from sketches is a process that requires several challenging problems including segmentation, 
recognition and reconstruction. Some of these tasks are harder for humans and some are harder for the machine. 
At the core of the problem lies the need for semantic understanding of the shape’s geometry from the sketch.  

% our approach

We present a semi-automatic approach for modelling man-made objects from a sketch by seperating tasks that are
difficult for a human from those that are difficult for a machine. The user assits recognition and segmentation
by choosing and placing specific primitives (cones, spheres, boxes, etc) in approximate locations on the sketch. The machine first snaps
the primitive to the sketch by fitting its projection to the sketch lines. Then it globally optimizes
the model by inferring geosemantic relationships (orthogonality, coplanarity, etc) the link the different parts and globally optimizing the model
to fit the sketch subject to constraints that mathematically describe those relationships. Those constraints ensure consistent 
look of the whole model, instead of being just a set of unrelated parts.

% primitives fitting

3D modelling is an interactive process and therefore the system should respond in real-time. However, non-convex constrained
numeric optimization is, potentially, a computationally expensive task. To close this gap we build a user interface
around providing good starting points for an iterative optimizaton procedure, allowing it to converge quickly and efficiently. We
use a two-stage approach where a primitive is first fit to the sketch without any constraints, and then its relationships
are inferred and a global-optimization procedure takes place.

% conclusion
In conclusion, we have created a sketch-based modelling software framework and provided a user interface that is both intuitive
for the user, and uses the framework in an efficient manner. It allows the user to create 3D models quickly and rapidly. 

